\subsubsection*{Thesis title}  
\Title

\subsubsection*{Abstract} 
A project that helps players during a game of 9-ball billiards by showing them how the white ball
would likely move after a shot. The system has two main parts: 1. Phone App on the Cue Stick - the app that
uses the phone’s motion sensors (like gyroscope and accelerometer) to track how the cue stick is held and
moved. The app sends this data in real time to a computer. 2. Camera Above the Table + Computer - A camera
above the billiard table sends live video to a computer. There, a program using e.g., OpenCV detects where all
the balls are based on their colors. It figures out where the cue ball is and updates the layout of the table.

\subsubsection*{Key words}  
Computer Vision, Billiards, Trajectory Prediction, Augmented Reality, Object Detection 

\subsubsection*{Tytuł pracy}
\begin{otherlanguage}{polish}
\TitleAlt
\end{otherlanguage}

\subsubsection*{Streszczenie} 
\begin{otherlanguage}{polish}
Projekt wspomagający graczy podczas gry w bilard (odmiana 9-bil) poprzez wizualizację przewidywanego toru ruchu białej bili po uderzeniu. System składa się z dwóch głównych części: 1. Aplikacja na kiju bilardowym -- wykorzystuje czujniki ruchu telefonu (takie jak żyroskop i akcelerometr) do śledzenia sposobu trzymania i ruchu kija. Aplikacja przesyła te dane w czasie rzeczywistym do komputera. 2. Kamera nad stołem i komputer -- kamera umieszczona nad stołem bilardowym przesyła obraz wideo na żywo do komputera. Następnie oprogramowanie (wykorzystujące m.in. OpenCV) wykrywa położenie wszystkich bil na podstawie ich kolorów, lokalizuje białą bilę i aktualizuje układ stołu.
\end{otherlanguage}

\subsubsection*{Słowa kluczowe} 
\begin{otherlanguage}{polish}
Wizja komputerowa, Bilard, Predykcja trajektorii, Rzeczywistość rozszerzona, Wykrywanie obiektów
\end{otherlanguage}

