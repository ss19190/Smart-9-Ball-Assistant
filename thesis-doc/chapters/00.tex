\subsubsection*{Thesis title}  
\Title

\subsubsection*{Abstract} 
The objective of this project is to help players during a game of 9-ball billiards by showing them how the white ball
would likely move after a shot. The system has two main parts: 1. Phone App on the Cue Stick --- the app that
uses the phone’s motion sensors (like gyroscope and accelerometer) to track how the cue stick is held and
moved. The app sends this data in real time to a computer. 2. Camera Above the Table + Computer --- A camera
above the billiard table sends live video to a computer. There, a program detects where all
the balls are based on their colors. It figures out where the cue ball is and updates the layout of the table. 
Both of those parts are essential, because accurate aiming without a good stroke is pointless. And same goes
for a great stroke without proper aiming. 

\subsubsection*{Key words}  
Computer Vision, Billiards, Trajectory Prediction, Augmented Reality, Object Detection 

\subsubsection*{Tytuł pracy}
\begin{otherlanguage}{polish}
\TitleAlt
\end{otherlanguage}

\subsubsection*{Streszczenie} 
\begin{otherlanguage}{polish}
Celem niniejszego projektu jest wspomaganie graczy podczas rozgrywki w bilard (odmiana 9-bil) poprzez wizualizację przewidywanego toru ruchu białej bili po uderzeniu. System składa się z dwóch głównych elementów: 1. Aplikacja mobilna na kiju bilardowym --- aplikacja wykorzystuje czujniki ruchu telefonu (takie jak żyroskop i akcelerometr) do śledzenia sposobu trzymania kija oraz jego ruchu. Aplikacja przesyła te dane w czasie rzeczywistym do komputera. 2. Kamera nad stołem i komputer --- kamera umieszczona nad stołem bilardowym przesyła obraz na żywo do komputera. Następnie program komputerowy wykrywa położenie wszystkich bil na podstawie ich kolorów, lokalizuje bilę białą i aktualizuje odwzorowanie układu stołu. Oba te elementy są kluczowe, ponieważ precyzyjne celowanie bez poprawnego technicznie uderzenia jest bezcelowe. Analogicznie, nawet idealne uderzenie nie przyniesie efektu bez właściwego wycelowania.
\end{otherlanguage}

\subsubsection*{Słowa kluczowe} 
\begin{otherlanguage}{polish}
Wizja komputerowa, Bilard, Predykcja trajektorii, Rzeczywistość rozszerzona, Wykrywanie obiektów
\end{otherlanguage}

